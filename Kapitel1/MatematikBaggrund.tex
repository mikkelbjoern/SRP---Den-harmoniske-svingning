\section{Matematisk baggrundsteori}
\subsection{Differentialligninger}
\newcommand{\LosEks}{$ \{ c\cdot e^{kx}|c\in \mR \} $}%Bruges i sætningen
Ved en differentialligning forstår vi en funktionnalligning, hvor funktionen minimum et sted er differentieret minimum en gang.
Ved en funktionalligning forstås en ligning, hvor det er en funktion, der er den ubekendte. 
Med en løsning til en differentialligning, refererer jeg fremover til en funktion, der gør differentialligningen sand. 
Ved den fuldstændige løsning til en differentialligning, refererer jeg fremover til mængden af \emph{alle} løsninger til differentialligningen. 
Nedenfor er et eksempel på en differentialligning. 
Denne er medtaget både som eksempel og for at gøre udregninger mere tilgængelige senere.

\begin{definition}
Ved en \textit{homogen} differentialligning i variablen $x$, forstås en differentialligning, som ikke har et konstantled (et led der ikke ændrer sig med $x$), hvis man tager alle led over på den samme side således at den anden side af lighedstegnet er lig $0$.
\end{definition}

\newcommand{\LDiff}[1]{a_{#1} \cdot \frac{d^{#1}y}{d x^{#1}}}
\begin{definition}
Ved en lineær differentialligning i $y$, forstås en ligning på formen:
$$\LDiff{n} + \LDiff{n-1} + \LDiff{n-2}+...+\LDiff{2}+\LDiff{1}=a_0$$
hvor $n$ er et naturligt tal, $y$ er en funktion af $x$, og alle $a$'erne er reelle ikke nødvendigvis forskellige konstanter og $a_n \neq 0$.
Vi siger da, at differentialligningen har orden $n$.

\end{definition}



\begin{thm}\label{thm:y'=k*y}
Lad $y$ være en differentialbel funktion og $k\in \mR$. Da har  differentialligningen $y' = ky$ den fuldstændige løsning \LosEks.
\end{thm}
\begin{proof}
Først vil jeg vise, at alle funktioner i den påståede fuldstændige løsning er løsninger til differentialligningen.
Lad $g$ være en funktion, der ligger i \LosEks .
Da er $g=c\cdot e^{kx}$ for et eller andet reelt tal $c$.
Jeg indsætter da løsningen i ligningen: 

\begin{align*}
y' &= ky\\
g' &= kg\\
(c\cdot e^{kx})' &= k\cdot c\cdot e^{kx}\\
k\cdot c\cdot e^{kx} &= k\cdot c\cdot e^{kx}\\
\end{align*}
Jeg ser da, at de to sider er lig hinanden, og dermed at $g$ er en løsning. 

Nu vil jeg vise, at hvis en funktion $h$ er en løsning til differentialligningen, da vil $h$ ligge i \LosEks.
Antag da netop, at $h$ er en løsning til differentialligningen. 
Dermed ved vi, at $h'=k\cdot h$.
Jeg vil nu lave omskrivninger på denne ligning. 
For at gøre disse omskrivninger nemmere, vil jeg definere en funktion $s=e^{-kx}$, som vi ved at differentiere indser opfylder, at $s'=k \cdot e^{-kx} = -ks$.
Ved omskrivning ses da:
\begin{align*}
h'&=k\cdot h \\
h'-kh &= 0\\
h's-ksh&=0 		& \text{(Ganger med }s\text{)}\\
h's+s'h&=0		& \text{(Ses ved $s'=-ks$)}\\
(hs)'&=0		&\text{(Ses ved kædereglen)}\\
hs+c_1 &= 0		&\text{(Ses ved at integrere, $c_1$ er integrationskonstant)}\\
h&= -c_1 \cdot \frac{1}{s}\\
h&= -c_1 \cdot e^{kx} &\text{(Ses ved $\frac{1}{s}=\frac{1}{e^{-kx}} = e^{kx}$)}
\end{align*}
Da ser vi, at $h$ må være på formen en konstant ($-c_1$) ganget med $e^{kx}$, hvilet netop karakteriserer elementer i \LosEks. 
Dermed ligger $h$ i \LosEks.
Da er vist, at hvis en funktion er en løsning til differentialligningen, ligger den også i \LosEks og det er også vist, at hvis et element ligger i \LosEks\text{,} da er det også en løsning. 
Dermed er \LosEks\text{} den fuldstændige løsning til differentialligningen. 
\end{proof}