\section{Nødvendig fysisk baggrundsteori}
\subsection{Hooks lov}\label{teori:Hooks lov}
Når man hiver i en fjeder, vil den have en kraft, modsatrettet den retning man hiver, der er proportional med hvor meget man har strukket fjederen\refFysA{286}. 
Dette giver anledning til formlen 
$$F =k\cdot x$$
hvor $x$ er den længde som fjederen er strukket, $F$ er kraften som fjederen trækker med og $k$ er proportionalitetskonstanten. 
$k$ kaldes i denne sammenhæng for fjederkonstanten.

\subsection{Newtons anden lov}\label{teori:Newtons anden lov}
Newtons anden lov fortæller ud fra den kraft et objekt påvirkes med og dets masse, hvor stor meget objektet vil accelerere. 
Den siger, at massen ganget med accelerationen af et objekt giver den kraft, som objektet bliver påvirket med\refFysA{217}.

\subsection{Vindmodstand}\label{teori:vindmodstand}
Når et objekt bevæger sig gennem luft eller en væske med en fart $v$, vil det have en gnidning med luften omkring sig, hvilket vil gøre farten mindre.
Ved et opslag i en formelsamling, ser man at kraften som vindmodstanden virker med, er proportional med farten i anden. 
Ved lave hastigheder er dette dog næsten det samme som at den er proportional med farten i første, og dette kan derfor være en okay antagelse af gøre sig i visse tilfælde\refMekanik{11}.
