\section{Forsøgsopsætning}
Hele denne opgave tager udgangspunkt i et forsøg.
I forsøget hænges et lod op i en fjeder der opfylder Hooks lov (se \ref{teori:Hooks lov}). 
Da tilføres loddet energi, således at det begynder at svinge. 
På figur \ref{fig:Basis Forsogsopsaetning} ses en skitse af forsøgets opsætning.

\begin{figure}[h]
\center
\opstilling{-1}%Se opstilling.sty i hovedmappen for definition. Ser bedst ud med udgangsposition i -1.

\caption{Skitse af forsøgsopsætning.}
\label{fig:Basis Forsogsopsaetning}
\end{figure} 
Man kan da overveje, hvilke kræfter der påvirker loddet. 
Det er klart at loddet er påvirket af en tyngdekraft. 
Derudover er loddet påvirket af kraften fra fjederen.
Til sidst vil loddets bevægelse også være påvirket af luften omkring den hvis loddet er sat i bevægelse. 

