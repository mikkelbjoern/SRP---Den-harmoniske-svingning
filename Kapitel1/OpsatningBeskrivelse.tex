\section{Forsøgsopsætning}\label{teori: opsatning af differentialligninger}
Hele denne opgave tager udgangspunkt i et forsøg.
I forsøget hænges et lod op i en fjeder der opfylder Hooks lov (se \ref{teori:Hooks lov}). 
Da tilføres loddet energi, således at det begynder at svinge. 
På figur \ref{fig:Basis Forsogsopsaetning} ses en skitse af forsøgets opsætning.

\begin{wrapfigure}{r}{0.4\textwidth}
\centering
\opstilling{-1}%Se opstilling.sty i hovedmappen for definition. Ser bedst ud med udgangsposition i -1.

\caption{Skitse af forsøgsopsætning.}
\label{fig:Basis Forsogsopsaetning}
\end{wrapfigure} 

\subsection{Teoretiske egenskaber for bevægelsen}
Man kan da overveje, hvilke kræfter der påvirker loddet. 
Det er klart at loddet er påvirket af en tyngdekraft. 
Derudover er loddet påvirket af kraften fra fjederen.
Til sidst vil loddets bevægelse også være påvirket af luften omkring den hvis loddet er sat i bevægelse. 
Jeg vil starte med at kigge på, hvordan loddet vil opføre sig hvis man ser på en forholdsvist langsom bevægelse i en kortere tidsperiode.
Da kan vi nemlig se bort fra luftens dæmpning. 

\subsubsection{Differentialligning uden dæmpning}\label{teori: Opstilling ligning uden dampning}
Hvis vi betragter nedadgående retning som værende positiv retning, kan vi altså opskrive den resulterende kraft på loddet som $F_{res} = -F_{fjeder}+F_{t}$, hvor $F_t$ betegner tyngdekraften på loddet. 
Vi bemærker dog at $F_t$ ikke er afhængig af af loddets stedfunktion. 
Dette gør den mere besværlig at arbejde med. 
Vi vil derfor sætte vores nulpunkt for kræfterne på en sådan måde at tyngdekraften bliver udlignet af at fjederen bliver strukket ud. 
Således har vi altså i stedet for $F_{res}=-F_{fjeder}$ eller $F_{res}+F_{fjeder}=0$.
Men vi kan da omskrive til en differentialligning ved at udnytte at $F_{fjeder}=k \cdot x(t)$ (se \ref{teori:Hooks lov}) hvor $x(t)$ er en funktion der beskriver loddets højde i forhold til et udgangspunkt (typisk vil dette udgangspunkt være den højde vil være i når det hænger stille).  
Derudover ved vi fra Newtons anden lov (se \ref{teori:Newtons anden lov}) at $F_{res}=m\cdot a(t) = m \cdot x''(t)$ 
(her er udnyttes at accelerationen af et objekt er lig dets stedfunktion differentieret dobbelt\refFysA{187}). 
Dermed får vi altså ligningen :
$$m\cdot x''(t)+k\cdot x(t)=0$$ 
som vi ser er en differentialligning i $x(t)$.

\subsubsection{Differentialligning med dæmpning}\label{teori: Opstilling af ligning med dampning}
Vi kan på samme måde som i forrige afsnit opskrive et udtryk der beskriver den resulterende kraft på loddet. 
Denne gang skal der bare trækkes et led mere fra på højresiden - nemlig luftmodstanden. 
Da vi i de fleste bevægelse tilnærmelsesvist kan beskrive luftmodstanden som værende proportional med hastigheden, kan vi skrive den som $- \mu x'(t)$ hvor $\mu$ er en proportionalitetskonstant og $x'(t)$ er stedfunktionen differentieret (og altså en beskrivelse af farten til tiden $t$\refFysA{187}).
her får vi altså en differentialligning på formen $m\cdot x''(t)=-k\cdot x(t)-\mu x'(t)$ eller omskrevet:
$$m\cdot x''(t)+\mu x'(t)+k\cdot x(t)=0$$
Fremover til $\mu x(t)$ blive omtalt som \textit{dæmpningsledet}, da det netop er luftmodstanden der gør at bevægelsen ikke bliver ved for evigt, da den giver anledning til en dæmpning. 