\section{Den komplekse eksponentialfunktion}
I gymnasiet bliver man kort introduceret til de komplekse tal, og jeg tager derfor for givet, at læseren kan løse en andengradsligning hvor diskriminanten er skarpt mindre end $0$ og derfor giver to komplekse løsninger. 
I gymnasiet lærer man dog ikke, at kunne opløfte tal i komplekse tal. 
Dette er dog en nødvendighed at kunne i denne opgave. 
Jeg vil derfor introducere definitionen af eksponentialfunktionen defineret på de komplekse tal $\mC$. 
Jeg vil ikke argumentere for, at denne opfører sig som den konventionelle definition med $\mR$, men bare lave en reference til en note fra DTU hvor dette kan ses. 

\begin{definition}[Den komplekse eksponentialfunktion]\label{def: Den komplekse eksponentialfunktion}

Lad $z=a+bi$ være et vilkårligt komplekst tal. Vi kalder da den komplekse eksponentialfunktion for $exp_\mC$. 
Vi definerer da denne til at være lig $exp_\mC (z)=e^{a} \cdot (cos(b) + sin(b)i)$.
Hvor $e$ er Eulers tal. 
\end{definition}

\begin{thm}\label{thm: expC = e^x}
$exp_\mC (x) = e^x$ hvis $x$ er et reelt tal. 
\end{thm}

\begin{proof}
Dette ses ret tydeligt ved at sætte $x+0i$ ind i definitionen af $exp_\mC$, der da vil blive $$exp_\mC(x) = e^x \cdot (cos(0)+sin(0)i)=e^x \cdot 1 = e^x \hspace{3cm}
\text{(Da $\sin (0)=0$ og $\cos (0)=1$)}$$
Dermed vist $exp_\mC (x) = e^x, \forall x \in \mR$
\end{proof}

\begin{thm}[Regneregler for $\exp_\mC$]\label{thm: regneregler for expC}
Følgende regneregler gælder for den komplekse eksponentialfunktion:
\begin{enumerate}
	\item $\eC (0)=1$
	\item $\eC (z_1+z_2) = \eC (z_1) \cdot \eC(z_2)$ for alle komplekse tal $z_1$ og $z_2$
	\item $(\eC (z))^n = exp_C (zn)$ for alle $n\in \mN$ og $z \in \mZ$.
\end{enumerate}
\end{thm}

\begin{proof}
Beviset for sætning \ref{thm: regneregler for expC} er ikke medtaget her grundet mangel på plads. Det kan læses i DTU's note om komplekse tal\refDTUKompleks{29.44}. 
\end{proof}

Vi kan se, at $\eC$ opfører sig på mange måder på samme måde som $e^x$, og specielt fordi $\eC$ altid tager de samme værdier som $e^x$ hvor den er defineret (de reelle tal, se sætning \ref{thm: expC = e^x}), tillader vi os fremover, at lade $e^x$ beskrive $\eC (x)$, og dermed er $e^x$ altså nu defineret fra $\mC$ over i $\mC$.