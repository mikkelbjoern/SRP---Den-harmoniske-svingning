\subsection{To komplekse rødder i karakterligningen}
\label{teori: Komplekse losninger i karakterligningen}
Jeg vil ikke føre fuldstændigt bevis for den fuldstændige løsning, når løsningerne til karakterligningen er komplekse. 
Dette gør jeg ikke fordi at beviset minder meget om beviset for sætning \ref{thm: fuldstandig losning}.

Man kunne formode ud fra sætning \ref{thm: fuldstandig losning}, at løsningerne ville være på formen $C\cdot e^{r_1 x} + D \cdot e^{r_2 x}$.
Her bliver vi dog nødt til først at bemærke noget om $r_1$ og $r_2$.
Vi ved at disse to er løsninger til den samme andengradsligning, og dermed må de også være komplekst konjugerede af hinanden\refKalkulus{133}. 
Dermed må vi kunne skrive dem som $r_1 = a + bi$ og $r_2 = a-bi$. 
Man kunne derfor tænke, at løsningen til den lineære andenordens differentialligning ville være på formen $C\cdot e^{(a+bi) x} + D \cdot e^{(a-bi)x}$.
Dette kan vi dog skrive om med definition \ref{def: Den komplekse eksponentialfunktion} til $Ce^{ax}(\cos(bx) + \sin(bx) i) + De^{ax}(\cos(-bx) + \sin(-bx) i)$.
Vi kan da flytte $e^{a}$ udenfor parentes og få 
$e^{ax}(C(\cos(bx) + \sin(bx) i) + D(\cos(-bx) + \sin(-bx) i))$.
Da det gælder om cosinus og sinus at $\cos(-x)=\cos(x)$ og $\sin(-x)=-\sin(x)$, kan vi videre omskrive til 4
$e^{ax}(C(\cos(bx) + \sin(bx) i) + D(\cos(bx) - \sin(bx) i))$. 
Dette kan ved brug af additionsformlen for sinus omskrives videre til $e^{ax}\cdot \sqrt{C^2 + D^2}\sin(bx+\phi)$. 
Jeg vil ikke føre et argument for den sidste omskrivning.
Dette skyldes at der kræves en forholdvist dybdegående forståelse for de trigonometriske funktioner for at kunne gennemføre argumentet. 
Argumentet for omskrivningen kan ses i Kalkulus\refKalkulus{538-539}.

Man kan argumentere for at mængden af alle funktioner på den før omtalte form er den fuldstændige løsning til den lineære andenordens homogene differentialligning når den har komplekse rødder i sin karakterligning. 
Jeg vil dog udelade beviset, da det føres på næsten samme måde som beviset for sætning \ref{thm: fuldstandig losning}. 

Jeg har altså den fuldstændige løsning til differentialligningen (når rødderne er komplekse) som:
$y(x) = e^{ax}\sqrt{C^2 + D^2}\sin(bx+\phi)$, i fysiksammehænge sætter vi dog ofte $\sqrt{C^2 + D^2}$ til at være lig med en konstant $A$, da denne viser sig at være en amplitude. 
Vi har derfor løsningerne på formen:
$$y(x) = e^{ax}A\sin(bx+\phi)$$


