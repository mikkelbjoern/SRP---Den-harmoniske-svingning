\section{Den andenordens lineære homogene differentialligning}
Det store problem i denne opgave er at bestemme den fuldstændige løsning til differentialligningerne der er beskrevet i afsnit \ref{teori: opsatning af differentialligninger}.
Disse er kan begge beskrives på formen $y'' + by' + cy = 0$, hvor $b,c \in \mR$ og $y$ er en differentialbel funktion fra de reelle tal over i de reelle tal. 
Bemærk her at der ikke er et konstantled foran $y''$. 
Dette gør ikke at vi mister generalitet, da vi altid ville kunne dividere hele ligningen igennem med den konstant der havde stået foran $y''$, og vi ville da have fået det på den ovenstående form. 
Jeg vil i dette afsnit bestemme den fuldstændige løsning til denne ligning. 
Mine udledninger her er stærkt inspirerede af Kalkulus\refKalkulus{529-539}. 

Jeg vil først vise et lemma der vil gøre mine udregninger nemmere senere.

\begin{lemma}
Hvis $g$ og $h$ er løsninger til den lineære andenordens differentialligning, da er $y(x) = C\cdot g(x) + D \cdot h(x)$ også en løsning for alle reelle konstanter $C,D$. 
\end{lemma}

\begin{proof}
Antag af $h$ og $g$ er løsninger til den lineære andenordens differentialligning. 
Vi starter med at indse at $y'=Cg' + Dh'$ og $y'' = Cg'' + Dh''$, ved brug af normale regler for differentiation. 

Vi betragter da ligningen med $y$ indsat:
\begin{align*}
y'' + by' + cy 	&= (Cg'' + Dh'') + b(Cg' + Dh') + c(Ch + Dh)	& \text{(Egenskaberne beskrevet ovenfor)}\\
				&= Cg'' + Dh'' + bCg' + bDh' + cCh + cDh		\\
				&= C(g'' + bg' + cg) + D(h'' + b'h + ch)		& \text{(Sætter udenfor parantes)}\\
				&= C\cdot 0 + D \cdot 0							& \text{(Udnytter at $g$ og $h$ er løsninger)}\\
				&= 0 
\end{align*}
Da kan vi ses at $y'' + by' + cy = 0$, dermed er $y$ er løsning til differentialligningen.
\end{proof}

Når vi snakker om lineære andenordens differentialligninger, er koefficienterne helt centrale for den fuldstændige løsning. 
Det vil vise sig senere, at en tilhørende andengradsligning til en lineær andenordens differentialligning, er helt central. Vi kalder denne for den \textit{Karakterligningen}.

\begin{definition}[Karakterligningen]
Karakterligningen for en lineær andenordens differentialligning $y'' + by' + cy = 0$, er ligningen $r^2 + br + c = 0$. 
Denne ligning har som bekendt en eller to komplekse løsninger $r_1 = \frac{-b + \sqrt{b^2 - 4c}}{2}$ og $r_2 = \frac{-b - \sqrt{b^2 - 4c}}{2}$. 
Fremover vil $r_1$ og $r_2$ referere til disse to definitioner af løsninger af karakterligningen for en given lineær andenordens differentialligning.

\end{definition}