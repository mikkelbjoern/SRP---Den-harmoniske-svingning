\section{Den andenordens lineære homogene differentialligning}
Det store problem i denne opgave er at bestemme den fuldstændige løsning til differentialligningerne der er beskrevet i afsnit \ref{teori: opsatning af differentialligninger}.
Disse er kan begge beskrives på formen $y'' + by' + cy = 0$, hvor $b,c \in \mR$ og $y$ er en differentialbel funktion fra de reelle tal over i de reelle tal. 
Bemærk her at der ikke er et konstantled foran $y''$. 
Dette gør ikke at vi mister generalitet, da vi altid ville kunne dividere hele ligningen igennem med den konstant der havde stået foran $y''$, og vi ville da have fået det på den ovenstående form. 
Jeg vil i dette afsnit bestemme den fuldstændige løsning til denne ligning. 
Mine udledninger her er stærkt inspirerede af Kalkulus\refKalkulus{529-539}. 

Jeg vil først vise et lemma der vil gøre mine udregninger nemmere senere.

\begin{lemma}\label{thm: Summen af to losninger er en losning}
Hvis $g$ og $h$ er løsninger til den lineære andenordens differentialligning, da er $y(x) = C\cdot g(x) + D \cdot h(x)$ også en løsning for alle reelle konstanter $C,D$. 
\end{lemma}

\begin{proof}
Antag af $h$ og $g$ er løsninger til den lineære andenordens differentialligning. 
Vi starter med at indse at $y'=Cg' + Dh'$ og $y'' = Cg'' + Dh''$, ved brug af normale regler for differentiation. 

Vi betragter da ligningen med $y$ indsat:
\begin{align*}
y'' + by' + cy 	&= (Cg'' + Dh'') + b(Cg' + Dh') + c(Ch + Dh)	& \text{(Egenskaberne beskrevet ovenfor)}\\
				&= Cg'' + Dh'' + bCg' + bDh' + cCh + cDh		\\
				&= C(g'' + bg' + cg) + D(h'' + b'h + ch)		& \text{(Sætter udenfor parantes)}\\
				&= C\cdot 0 + D \cdot 0							& \text{(Udnytter at $g$ og $h$ er løsninger)}\\
				&= 0 
\end{align*}
Da kan vi ses at $y'' + by' + cy = 0$, dermed er $y$ er løsning til differentialligningen.
\end{proof}

Når vi snakker om lineære andenordens differentialligninger, er koefficienterne helt centrale for den fuldstændige løsning. 
Det vil vise sig senere, at en tilhørende andengradsligning til en lineær andenordens differentialligning, er helt central. Vi kalder denne for den \textit{karakterligningen}.

\begin{definition}[Karakterligningen]
Karakterligningen for en lineær andenordens differentialligning $y'' + by' + cy = 0$, er ligningen $r^2 + br + c = 0$. 
Denne ligning har som bekendt en eller to komplekse løsninger $r_1 = \frac{-b + \sqrt{b^2 - 4c}}{2}$ og $r_2 = \frac{-b - \sqrt{b^2 - 4c}}{2}$. .
\end{definition}

Jeg vil i denne udledning kigge på to tilfælde af karakterligningen. 
Der hvor den har to komplekse løsninger og der hvor den har to reelle.
Grunden til at jeg ikke kigger på der hvor den har en reel rod, er at det kræver at $b^2 = 4c$ (da dette gør diskriminanten til $0$), hvilket er et særligt specialtilfælde, som meget sjældent vil forekomme i fysikken.

\subsection{To reelle løsninger af karakterligningen}
Jeg vil nu kigge på løsninger til den lineære andenordens differentialligning. 
Først indser jeg at $e^{rx}$ er en løsning til den lineære andenordens differentialligning hvis ellers $r$ er en løsning til den karakterligningen.

\begin{thm}\label{thm: e^rx er en losning}
Lad en lineær andenordens differentialligning være på formen $y'' + by' + cy = 0$ og lad $r$ være en løsning til karakterligningen for denne.
Da er $e^{rx}$ en løsning til differentialligningen. 
\end{thm}

\begin{proof}
Jeg ser først på funktionen differentieret og dobbelt differentieret. Begge disse kan findes ved normale regler for differentiation: 
$(e^{rx})' = r \cdot e^{rx}$ og 
$(e^{rx})'' = r\cdot (e^{rx})' = r^2 \cdot e^{rx}$.

Jeg indsætter nu funktionen i ligningen:
\begin{align*}
y'' + by' + cy 	&= (e^{rx})'' + b\cdot (e^{rx})' + c\cdot e^{rx} \\
				&= r^2 \cdot e^{rx} + br\cdot e^{rx} + c e^{rx} \\
				&= e^{rx} (r^2 + br + c) \\
				&= e^{rx} \cdot 0 		& \text{(Da $r$ er en løsning til karakterligningen)}\\
				&=0
\end{align*}

Da er det vist at $y=e^{rx}$ får differentialligningen til at blive $0$ og at dette derfor er en løsning. 
\end{proof}

\begin{thm}
$y = C \cdot e^{r_1 x} + D \cdot e^{r_2 x}$ er en løsning til differentialligningen $y'' + by' + c = 0$ hvis $r_1$ og $r_2$ er reelle løsninger til dennes karakterligning, hvor $C,D$ er vilkårlige reelle tal.
\end{thm}
\begin{proof}
Vi ser først at $e^{r_1 x}$ og $e^{r_2 x}$ begge hver for sig er løsninger ved brug af sætning \ref{thm: e^rx er en losning}.
Derefter kan vi bruge sætning \ref{thm: Summen af to losninger er en losning} til at se at disse to løsninger ganget med vilkårlige konstanter også er en løsning.
Dermed må $y = C \cdot e^{r_1 x} + D \cdot e^{r_2 x}$ være en løsning til differentialligningen. 
\end{proof}