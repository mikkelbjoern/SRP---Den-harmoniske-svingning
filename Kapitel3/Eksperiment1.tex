\section{Forsøg: Svingning af lod i luft i kort tid}\label{exp1: Forsog 1 - hele afsnittet}

\subsection{Forsøgsbeskrivelse}\label{exp1: Beskrivelse af experiment}
\subsubsection{Materialer}
\begin{wrapfigure}{r}{0.4\textwidth}
\centering
\opstillingEt{-1}%Se opstilling.sty i hovedmappen for definition. Ser bedst ud med udgangsposition i -1.

\caption{Skitse af forsøgsopsætning til forsøg 1. Den grå figur under loddet er en ultralydsensor.}
\label{fig:Forsogsopsaetning 1}
\end{wrapfigure} 

Disse materialer blev brugt til forsøget:
\begin{itemize}
	\setlength\itemsep{-1em}
	\item Lod
	\item Fjeder
	\item Kraftmåler
	\item Tavlelineal ($1$ meter)
	\item Stativ til opsætning
	\item Ultralydsensor
	\item Computer til opsamling af data
\end{itemize}

\subsubsection{Udførelse}\label{exp1: Udforelse}



Før selve udførslen af forsøget, bestemmer jeg en fjederkonstant for den fjeder jeg arbejder med. 
Dette gøres ved at måle samhørende værdier at udstrækning af fjederen og hvor stor en kraft den trækker med. 

I dette forsøg har jeg opsat et lod i en fjeder, som set på figur \ref{fig:Forsogsopsaetning 1}.
Derefter har jeg sat loddet i svingninger og målt loddets position med ultralydsensoren i en periode på $10$ sekunder. 
Jeg gentog dette forsøg 10 gange. 


\subsection{Hypotese}\label{exp1: Hypotese}
Da dette forsøg forløber over forholdsvist kort tid, forventer jeg at man betragte denne bevægelse som om den ikke er dæmpet, da dæmpningsledet vil være forholdvist småt. 
Dermed forventes differentialligningen 
$$m\cdot x'' = -k \cdot x$$
at være opfyldt, jvf. afsnit \ref{teori: Opstilling ligning uden dampning}.
Dette kan omskrives til $m\cdot x'' + k\cdot x=0$, hvilket vi ser er en andenordens lineær homogen differentialligning. 
Vi kan derfor bestemme en karakterligning som $mr^2 + k = 0 \Rightarrow r^2 + \frac{k}{m} = 0$ og dermed får vi $r = \frac{\pm \sqrt{-4\frac{k}{m}}}{2}=\pm\sqrt{\frac{k}{m}}i$.
Dermed skal vi have fat i den løsning af differentialligningen hvor løsningerne til karakterligningen er komplekse. 
Vi får da at løsningen skal være på formen (se afsnit \ref{teori: Komplekse losninger i karakterligningen}:
$$y(x) = e^{ax}A\sin(bx+\phi) \text{ hvor rødderne i karakterligningen er } r = a \pm bi$$
Vi ser dog her at rødderne i vores karakterligning ikke har nogen realdel, og dermed får vi at $e^{ax}=e^{0x}=1$.
Da $b=\sqrt{\frac{k}{m}}$ forventer vi derfor at bevægelsen vil kunne beskrives som 
$$x(t)=A\sin (\frac{k}{m}t+\phi)$$
hvor $\phi$ er en eller anden konstant. 


\subsection{Data}\label{exp1: Data}
\subsubsection{Svingning}
Data til dette forsøg er vedlagt som bilag i filen ''Forsøg 1 - Data.pdf''
Denne fil indeholder alt rådata opsamlet af ultralydsensoren i de 10 forsøg (OBS. filen er $130$ sider lang da der er blevet opsamlet $40$ målinger i sekundet). 
Forsøg nummer 10 er placeret først grundet måden mit dataopsamlingsprogram fungerer på.

\subsubsection{Fjederkonstant}\label{exp1: Fjederkonstant}
Ydermere er der vedlagt en fil ved navn ''Forsøg 1 og 2 fjederkonstant - Data.pdf''.
Denne fil indeholder data til bestemmelse af fjederkonstanten. 

\subsubsection{Masse af fjeder og lod}\label{exp1: Masse af fjeder og lod}
Jeg har målt massen af lodet og af fjederen i forbindelse med forsøget. 
Disse to er nedskrevet i tabel \ref{tabel: Masser forsog 1}.
\begin{table}[h]
\centering
\begin{tabular}{|c|c|}
\hline 
 & Masse ($g$) \\ 
\hline 
Fjeder & $77.09$ \\ 
\hline 
Lod & $32.45$ \\ 
\hline 
\end{tabular} 
\caption{Relevante masser til forsøg 1.}
\label{tabel: Masser forsog 1}
\end{table}

\subsection{Databehandling}\label{exp1: databehandling afsnit}
\subsubsection{Fjederkonstant}\label{databehandling: tyk fjeder fjederkonstant}
For at finde en fjederkonstant for fjederen der bliver brugt til forøget, har jeg taget de sammenhørende værdier af fjederkraft og udstrækning og plottet dem i kraft som afhængig og udstrækning som uafhængig variabel. 
Derefter har jeg lavet en best-fit lineær sammenhæng. 
Denne havde en forholdsvist lav RMSE på $0.044N$, og derfor virker dataen pålidelig. 
Vi må da have fjederkonstanten som hældningen på dette plot, i dette tilfælde $3.6\frac{N}{m}$.

Den tilhørende graf samt best-fit plot kan ses i bilaget "Forsøg 1 - grafer.pdf". 
 

\subsubsection{Best-fit kurver}\label{exp1: Best-fit kurver}
For at tjekke hvor godt hypotesen holder, vil jeg forsøge at lægge få min computer til at lave en best-fit kurve udfra funktionen $x(t)=A\sin (\sqrt{\frac{k}{m}}+\phi) + s_0$.
Det tilføjede $s_0$ for enden skyldes at loddet har hængt en afstand over ultralydsensoren som der skal justeres for. 

I tabel \ref{tabel: bestfitkurver forsog 1} kan ses værdier for $A, \sqrt{\frac{k}{m}}, \phi ,s_0$ og så den gennemsnitlige afvigelse målt i meter (RMSE).


\begin{table}[h]
\centering
\begin{tabular}{|l|c|c|c|c|c|}
\hline
\textbf{}          & \textbf{A}($m$) & \textbf{$\sqrt{\frac{k}{m}}$}($s^{-1}$) & \textbf{$\phi$} & \textbf{$s_0$}($m$) & \textbf{RMSE }($m$) \\ \hline
\textbf{Forsøg 10} & 0,08854    & 7,73                          & 584,8           & 0,2929         & 0,003136      \\ \hline
\textbf{Forsøg 1}  & 0,04319    & 7,632                         & 6,99            & 0,2984         & 0,001839      \\ \hline
\textbf{Forsøg 2}  & 0,06744    & 7,669                         & 20,9            & 0,2963         & 0,001766      \\ \hline
\textbf{Forsøg 3}  & 0,07559    & 7,692                         & 38,58           & 0,2948         & 0,002061      \\ \hline
\textbf{Forsøg 4}  & 0,081      & 7,707                         & -1,786          & 0,2933         & 0,003476      \\ \hline
\textbf{Forsøg 5}  & 0,08944    & 7,734                         & -1,145          & 0,2936         & 0,004427      \\ \hline
\textbf{Forsøg 6}  & 0,08174    & 7,709                         & 3,467           & 0,294          & 0,00381       \\ \hline
\textbf{Forsøg 7}  & 0,06552    & 7,663                         & 818,2           & 0,2954         & 0,001818      \\ \hline
\textbf{Forsøg 8}  & 0,05712    & 7,648                         & 9,939           & 0,2965         & 0,001589      \\ \hline
\textbf{Forsøg 9}  & 0,04544    & 7,633                         & 9,7             & 0,297          & 0,001462      \\ \hline
\end{tabular}

\caption{De forskellige værdier på best-fit kurver. De tilhørende grafer til dataen samt best-fit-plot er vedlagt som bilag i filen ''Forsøg 1 - grafer.pdf''}
\label{tabel: bestfitkurver forsog 1}
\end{table}

\subsubsection{Sammenligning af værdier af $\sqrt{\frac{k}{m}}$}\label{exp1: teoretisk bestemmelse af sqrt k over m}
For at tjekke om konstanter tager de forventede værdier, vil jeg lave en sammenligning af værdien for $\sqrt{\frac{k}{m}}$ fundet ved best-fit-kurven og ved måling på fjeder og lod.

Værdierne fra best-fit-kurven der kan ses i tabel \ref{tabel: bestfitkurver forsog 1}, er meget ens, og jeg vil derfor bare bruge et gennemsnit af dem. 
Dette gennemsnit bliver da ca. $7.68s^{-1}$.

Massen af loddet og fjederkonstanten har jeg begge bestemt individuelt. 
Fjederkonstanten kender jeg som $3.6\frac{N}{m}$, hvilket ses længere oppe i dette afsnit. 
Massen er lidt mindre ligetil. 
Vi kender godt nok massen af loddet fra tabel \ref{tabel: Masser forsog 1}. 
Der er dog det problem, at det ikke kun er loddet der svinger i bevægelsen, men også en del af fjederen. 
I teorien er det ca. $\frac{1}{3}$ af fjederens masse der skal regnes med i bevægelsen\refFysA{300}.
Vi kan derfor lave en masse $m=m_{lod}+\frac{1}{3}\cdot m_{fjeder}=32.45g+\frac{77.09g}{3}=58.15g$. 
Dermed kan vi bestemme en værdi for $\sqrt{\frac{k}{m}}$ som:
$$\sqrt{\frac{k}{m}}=\sqrt{\frac{3.6\frac{N}{m}}{58.15g}}=\sqrt{0.062\frac{\frac{N}{m}}{g}}=\sqrt{0.062\frac{1000\frac{g\cdot m}{s^2\cdot m}}{g}}=\sqrt{62s^{-2}}=7.87s^{-1}$$

Disse to værdier ser ud til at stemme rimelig godt overens, da de har en forholdsvist lille differens. 



\subsubsection{Vurdering af data}\label{exp1: Vurdering af data}
Det første der er værd at bemærke, er at værdierne af $\sqrt{\frac{k}{m}}$ ligger meget tæt på hinanden. 
Dette stemmer godt overens med teorien da denne værdi ikke bør være afhængig af hvor meget der energi der bliver tilførst systemet til at starte med, men kun hvilken masse lodet har og hvilken fjederkonstant fjederen har. 
Da forsøget er lavet med samme fjeder og samme lod, må man forvente samme værdi, hvilket også er det vi observerer, da der kun er en forskel på $7.734s^{-1}-7.632s^{-1}=0.102s^{-1}$ på den største og mindste værdi, hvilket er relativt lidt i forhold til de ca. 7 en halv tallene ligger på.



Derudover er det værd at bemærke de meget lave afvigelser. 
Den gennemsnitlige afvigelse ligger imellem $0.3136cm$ og $0.4427cm$ i forhold til amplituder på helt op til $8.944cm$. 
Dette må anses som forholdsvist små afvigelser i forhold til hele bevægelsen.


\subsection{Konklusion}


