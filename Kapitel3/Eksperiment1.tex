\pagebreak
\section{Forsøg: Svingning af lod i luft i kort tid}

\subsection{Forsøgsbeskrivelse}
\subsubsection{Materialer}
\begin{wrapfigure}{r}{0.4\textwidth}
\centering
\opstillingEt{-1}%Se opstilling.sty i hovedmappen for definition. Ser bedst ud med udgangsposition i -1.

\caption{Skitse af forsøgsopsætning til forsøg 1. Den grå figur under loddet er en ultralydsensor.}
\label{fig:Forsogsopsaetning 1}
\end{wrapfigure} 

Disse materialer blev brugt til forsøget:
\begin{itemize}
	\setlength\itemsep{-1em}
	\item Lod
	\item Fjeder
	\item Kraftmåler
	\item Tavlelineal ($1$ meter)
	\item Stativ til opsætning
	\item Ultralydsensor
	\item Computer til opsamling af data
\end{itemize}

\subsubsection{Udførelse}



Før selve udførslen af forsøget, bestemmer jeg en fjederkonstant for den fjeder jeg arbejder med. 
Dette gøres ved at måle samhørende værdier at udstrækning af fjederen og hvor stor en kraft den trækker med. 

I dette forsøg har jeg opsat et lod i en fjeder, som set på figur \ref{fig:Forsogsopsaetning 1}.
Derefter har jeg sat loddet i svingninger og målt loddets position med ultralydsensoren i en periode på $10$ sekunder. 
Jeg gentog dette forsøg 10 gange. 


\subsection{Hypotese}
Da dette forsøg forløber over forholdsvist kort tid, forventer jeg at man betragte denne bevægelse som om den ikke er dæmpet, da dæmpningsledet vil være forholdvist småt. 
Dermed forventes differentialligningen 
$$m\cdot x'' = -k \cdot x$$
at være opfyldt, jvf. afsnit \ref{teori: Opstilling ligning uden dampning}.
Dette kan omskrives til $m\cdot x'' + k\cdot x=0$, hvilket vi ser er en andenordens lineær homogen differentialligning. 
Vi kan derfor bestemme en karakterligning som $mr^2 + k = 0 \Rightarrow r^2 + \frac{k}{m} = 0$ og dermed får vi $r = \frac{\pm \sqrt{-4\frac{k}{m}}}{2}=\pm\sqrt{\frac{k}{m}}i$.
Dermed skal vi have fat i den løsning af differentialligningen hvor løsningerne til karakterligningen er komplekse. 
Vi får da at løsningen skal være på formen (se afsnit \ref{teori: Komplekse losninger i karakterligningen}:
$$y(x) = e^{ax}A\sin(bx+\phi) \text{ hvor rødderne i karakterligningen er } r = a \pm bi$$
Vi ser dog her at rødderne i vores karakterligning ikke har nogen realdel, og dermed får vi at $e^{ax}=e^{0x}=1$.
Da $b=\sqrt{\frac{k}{m}}$ forventer vi derfor at bevægelsen vil kunne beskrives som 
$$x(t)=A\sin (\frac{k}{m}t+\phi)$$
hvor $\phi$ er en eller anden konstant. 


\subsection{Data}

\subsection{Databehandling}

\subsection{Konklusion}


