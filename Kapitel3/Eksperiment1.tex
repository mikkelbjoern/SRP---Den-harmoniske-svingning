\pagebreak
\section{Forsøg: Svingning af lod i luft i kort tid}

\subsection{Forsøgsbeskrivelse}
\subsubsection{Materialer}
\begin{wrapfigure}{r}{0.4\textwidth}
\centering
\opstillingEt{-1}%Se opstilling.sty i hovedmappen for definition. Ser bedst ud med udgangsposition i -1.

\caption{Skitse af forsøgsopsætning til forsøg 1. Den grå figur under loddet er en ultralydsensor.}
\label{fig:Forsogsopsaetning 1}
\end{wrapfigure} 

Disse materialer blev brugt til forsøget:
\begin{itemize}
	\setlength\itemsep{-1em}
	\item Lod
	\item Fjeder
	\item Kraftmåler
	\item Tavlelineal ($1$ meter)
	\item Stativ til opsætning
	\item Ultralydsensor
	\item Computer til opsamling af data
\end{itemize}

\subsubsection{Udførelse}



Før selve udførslen af forsøget, bestemmer jeg en fjederkonstant for den fjeder jeg arbejder med. 
Dette gøres ved at måle samhørende værdier at udstrækning af fjederen og hvor stor en kraft den trækker med. 

I dette forsøg har jeg opsat et lod i en fjeder, som set på figur \ref{fig:Forsogsopsaetning 1}.
Derefter har jeg sat loddet i svingninger og målt loddets position med ultralydsensoren i en periode på $10$ sekunder. 
Jeg gentog dette forsøg 10 gange. 


\subsection{Hypotese}

\subsection{Databehandling}

\subsection{Konklusion}


