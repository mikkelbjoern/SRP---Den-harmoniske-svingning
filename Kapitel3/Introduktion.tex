\chapter{Forsøg}
Hele denne opgave tager udgangspunkt i et forsøg.
Jeg vil her beskrive dette forsøg, og udnytte den matematik, jeg har beskrevet, til at finde ud af om de opstillede differentialligninger fra afsnit \ref{teori: opsatning af differentialligninger} rent faktisk passer på virkeligheden.

Jeg har udført to forsøg, som hvert i sær er tænkt til at afprøve et stykke teori:
\begin{itemize}
	\item Svingning af lod i luft i kort tid
	\item Svingning af lod i luft i længere tid
\end{itemize}

Det første forsøg er ment til at skulle passe på differentialligningen uden dæmpningsled.

Det andet forsøg er tænkt til at passe med et dæmpningsled ved lav hastighed, og altså derfor med dæmpningsledet i 1. potens.
Dette forsøg undersøger også om en model med dæmpningsledet i anden potens vil fungere bedre, end en med leddet i første potens.  
