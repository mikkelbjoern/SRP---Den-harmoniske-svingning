\section{Opsummering af forsøg 1 og 2}
I både forsøg 1 og 2 har det være muligt at bruge den andenordens lineære homogene differentialligningsmodel til at beskrive bevægelsen. 
Det viser sig, at den tid som loddet svinger, er det der betyder mest for, om dæmpningsledet skal tages i betragtning. 
Det viser sig også at tiden lodet svinger har betydning for om modellen kan bruges. 
Ved de forholdsvist lave amplituder som jeg har studeret i forsøg 2, har det vist sig at modellen bliver rimelig upræcis efter ca. 40 sekunder. 

Hvis modellen skal kunne bruges over en længere tidsperiode, blive man nødt til at bruge en numerisk løsning af den ikke lineære model som er beskrevet tidligere.
I denne forbindelse vil det være optimalt at bestemme $\mu$ ud fra mere specifik teori om vindmodstand, da størrelsen ikke ser ud til at kunne bestemmes entydigt ud fra disse forsøg, grundet de varierende værdier af $a$ i tabel \ref{tabel: best fit exp2}.

Dog er modellen meget stærk og får lave afvigelser eksperimentelt, derfor ser differentialligningerne ud til at være et stærkt værktøj til at beskrive modellen. 
