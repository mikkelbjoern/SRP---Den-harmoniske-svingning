\section{Forsøg: Svingning af lod i luft i længere tid}

\subsection{Forsøgsbeskrivelse}
Dette forsøg forløber nøjagtigt på samme måde som forsøg 1 (se forsøgsbeskrivelse \ref{exp1: Beskrivelse af experiment}).
Der er dog den forskel at forsøget forløber over $50$ sekunder.

Jeg gør brug af samme fjeder og lod som i forsøg 1.

\subsection{Hypotese}
Da dette forsøg forløber over længere tid end det forrige forsøg, forventes det at vindmodstanden vil spille en markant rolle. 
Derfor vil vi forvente at det følger en anden differentialligning hvor dæmpningsledet er med, nemlig:
$m\cdot x''= -\mu \cdot x' - k\cdot x$
eller omskrevet til den form som vi har behandlet matematisk:
$$x''+ \frac{\mu}{m} \cdot x' + \frac{k}{m}\cdot x=0$$
Dette kan vi  se er en andenordens homogen lineær differentialligning.
Vi kan derfor opskrive en karakterligning for den:
$r^2 + \frac{\mu}{m} r + \frac{k}{m} = 0$, med rødderne $r = \dfrac{-\frac{\mu}{m} \pm \sqrt{\frac{\mu^2}{m^2}-4\frac{k}{m}}}{2}$.
Vi vil nu gerne bruge disse rødder til at opskrive en forventet funktion til beskrivelse af bevægelsen. 
Dog har vi det problem at vi ikke ved om rødderne bliver komplekse eller reelle, da dette afhænger af hvor stor en værdi $\mu$ er i forhold til $k$.
Hvis vi har at størrelsen $\frac{\mu^2}{m^2}-4\frac{k}{m}$ er positiv, så må vi have reelle rødder, hvorimod vi må have komplekse hvis den er negativ. 
Der er selvfølgelig også grænsetilfældet hvor det er lig 0, men da dette er højest usandsynligt vil jeg ikke behandle dette. 

\subsection{Data}

\subsection{Databehandling}

\subsection{Konklusion}


