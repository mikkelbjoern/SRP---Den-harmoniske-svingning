\section{Forsøg: Svingning af lod i luft i længere tid}

\subsection{Forsøgsbeskrivelse}
Dette forsøg forløber nøjagtigt på samme måde som forsøg 1 (se forsøgsbeskrivelse \ref{exp1: Beskrivelse af experiment}).
Der er dog den forskel at forsøget forløber over $50$ sekunder og at forsøget kun er gentaget 5 gange. 

Jeg gør brug af samme fjeder og lod som i forsøg 1.

\subsection{Hypotese}\label{exp2: Hypotese}
Da dette forsøg forløber over længere tid end det forrige forsøg, forventes det at vindmodstanden vil spille en markant rolle. 
Derfor vil vi forvente at det følger en anden differentialligning hvor dæmpningsledet er med, nemlig:
$m\cdot x''= -\mu \cdot x' - k\cdot x$
eller omskrevet til den form som vi har behandlet matematisk:
$$x''+ \frac{\mu}{m} \cdot x' + \frac{k}{m}\cdot x=0$$
Dette kan vi  se er en andenordens homogen lineær differentialligning.
Vi kan derfor opskrive en karakterligning for den:
$r^2 + \frac{\mu}{m} r + \frac{k}{m} = 0$, med rødderne $r = \dfrac{-\frac{\mu}{m} \pm \sqrt{\frac{\mu^2}{m^2}-4\frac{k}{m}}}{2}=
-\frac{\mu}{2m}  \pm   \frac{\sqrt{-\frac{\mu^2}{m^2}+4\frac{k}{m}}}{2}$.
Vi vil nu gerne bruge disse rødder til at opskrive en forventet funktion til beskrivelse af bevægelsen. 
Dog har vi det problem at vi ikke ved om rødderne bliver komplekse eller reelle, da dette afhænger af hvor stor en værdi $\mu$ er i forhold til $k$.
Hvis vi har at størrelsen $\frac{\mu^2}{m^2}-4\frac{k}{m}$ er positiv, så må vi have reelle rødder, hvorimod vi må have komplekse hvis den er negativ. 
Der er selvfølgelig også grænsetilfældet hvor det er lig $0$, men da dette er højest usandsynligt vil jeg ikke behandle dette. 

Da vi skal opstille en hypotese, vil det være forventeligt at rødderne er komplekse. 
Dette fordi vi i forsøg 1 gjorde præcis det samme som her og så at en model med komplekse rødder så ud til at fungere. 

I det vores løsninger til karakterligningen forventes at være komplekse, kan vi da forvente en bevægelse der kan beskrives ved:
$$e^{-\frac{\mu}{2m} \cdot x}A\sin(\frac{\sqrt{-\frac{\mu^2}{m^2}+4\frac{k}{m}}}{2} \cdot x+\phi)$$
\subsection{Data}
Rådata til dette forsøg er vedlagt i filen ''Forsøg 2 - Data.pdf'' og data til bestemmelse af fjederkonstanten ligger i filen ''Forsøg 1 og 2 fjederkonstant - Data.pdf''. 

Dataen er præsenteret på præcis samme måde som i forsøg 1, se derfor afsnit \ref{exp1: Best-fit kurver} for forklaring af dataopsætning.



\subsection{Databehandling}
\subsubsection{Bestemmelse af fjederkonstant}
Da fjederen er den samme som i forsøg 1, er fjederkonstanten bestemt på samme måde som i afsnit \ref{exp1: databehandling afsnit}, og er derfor $3.6\frac{N}{m}$.

For at tjekke om dataen rent faktisk opfører sig som opfører sig som forventet, vil jeg forsøge at lave best-fit kurver på formen:
$$e^a \cdot A \cdot \sin (bx+\phi)+s_0$$
Her står $a$ i stedet for konstanten $-\frac{\mu}{2m}$ og $b$ i stedet for 
$\frac{\sqrt{-\frac{\mu^2}{m^2}+4\frac{k}{m}}}{2}$.
Desuden er $s_0$ lagt til for at tage højde for, at loddet hænger et stykke over sensoren.

\begin{table}[h]
\centering
\begin{tabular}{|l|l|l|l|l|l|l|}
\hline
\multicolumn{1}{|c|}{} & \multicolumn{1}{c|}{\textbf{$a(s^{-1})$}} & \multicolumn{1}{c|}{\textbf{$A(m)$}} & \multicolumn{1}{c|}{\textbf{$b(s^{-1})$}} & \multicolumn{1}{c|}{\textbf{$\phi$}} & \multicolumn{1}{c|}{\textbf{$s_0(m)$}} & \multicolumn{1}{c|}{\textbf{RMSE$(m)$}} \\ \hline
\textbf{Forsøg A}      & -0,05858                          & 0,05713                           & 7,379                             & 30,21                                & 0,2955                              & 0,0009385                          \\ \hline
\textbf{Forsøg B}      & -0,06614                          & 0,06734                           & 7,384                             & 2,985                                & 0,2952                              & 0,003745                           \\ \hline
\textbf{Forsøg C}      & -0,08001                          & 0,09877                           & 7,395                             & 35,1                                 & 0,2949                              & 0,004438                           \\ \hline
\textbf{Forsøg D}      & -0,06506                          & 0,06517                           & 7,377                             & 7,275                                & 0,2958                              & 0,0009644                          \\ \hline
\textbf{Forsøg E}      & -0,06358                          & 0,06068                           & 7,385                             & 6,25                                 & 0,2959                              & 0,0007398                          \\ \hline
\end{tabular}
\caption{Konstanter fra best-fit-kurver til forsøg 2. De tilhørende grafer er vedlagt som bilag i filen ''Forsøg 2 - grafer.pdf''.}
\end{table}

\subsubsection{Vurdering af data}

Som udgangspunkt ser det meget godt ud med de meget lave værdier af RMSE. 
Dog kan man se et problem hvis man kigger på grafen.
Det ser ud til at der ikke kan findes en passende værdi af $a$, således at best-fit kurven passer på kurven hele vejen igennem. 
Dette kan ses meget tydeligt ved at kigge på den sidste del af grafen, hvor best-fit-kurven er mindre end den målte data.
der kan ses et eksempel på dette på figur \ref{fig: Fejlkurve exp2}.

\begin{figure}[h]
\center
\includegraphics[scale=0.5]{Figurer/Kurvefejl}
\caption{Indzoomet udklik af de sidste ca. $4.5$ sekunder af forsøg E. Det ses her tydeligt at dataen (rød) har et markant større udsving end best-fit-kurven (sort).}
\label{fig: Fejlkurve exp2}
\end{figure}

Grunden til den forskel som ses på figur \ref{fig: Fejlkurve exp2}, er formentlig af vores model for vindmodstanden ikke er helt god nok. 
Med den teori vi har om vindmodstand fra afsnit \ref{teori:vindmodstand}, vil det være et logisk gæt at sige at vi skal have en anden eksponent på vindmodstanden end 1. 
Dette vil kan man forsøge på ved at lave en numerisk løsning til differentialligningen. 





\subsection{Konklusion}


