\chapter{Perspektivering - Affjedring af biler}
Den harmoniske svingning optræder mange andre steder i mekanikken end bare når et lod svinger i en fjeder.
Det ses blandt i forbindelse med penduler og generet med ting der drejer rundt.
Derudover er det også relevant at kigge på i forbindelse med større bygninger som eksempelvis broer, der kan gå i svingninger hvis de ikke er konstrueret rigtigt. 

Et andet sted hvor man kan se svingninger af fjedre meget markant er i forbindelse med affjedring af biler. 
Det er nemlig sådan at en bils hjul er forbundet til fjedre. 
En skitse af dette kan ses på figur \ref{fig: Affjedring af bil}.


Fjederen er til for at sørge for at bilen hele tiden fastholder kontakt med vejen - også selvom den kører over en ujævnhed i vejen. 
Hvis fjederen går i stykker, kan det være meget problematisk. 
Der kan ske det, at fjederen dæmper for langsomt og at man derfor oplever at bilen hopper.
En video af af dette kan ses på YouTube\refBilVideo.
Som det ses på videoen, så bliver bilen ved med at svinge efter at manden har givet slip. 
Hvis man betragter denne svingning med den teori der er beskrevet i denne opgave, så svarer bilen til vores lod.
Vi er da interesserede i at få svingningen af bilen til at dø ud så hurtigt som muligt. 
Her arbejder vi ikke med luftmodstand, men derimod med en mekanisk dæmpning. 
Hvis vi antager at svingningen dæmpes på en måde, således at dæmpningen er proportional med hastigheden af svingningen med en proportionalitetskonstant $\xi$, har vi igen en lineær andenordens homogen differentialligning

\begin{equation}
mx''(t)+\xi x'(t) + k x(t) = 0
\end{equation}

hvor $k$ er fjederkonstanten for bilens fjeder, $m$ er bilens masse og $x$ er en funktion af tiden der beskriver hvor højt/lavt bilen befinder sig i forhold til sit udgangspunkt.

Da vi er interesserede i at bilen hopper så lidt som muligt, er vi ikke interesserede i en harmonisk svingning. 
Vi ved fra teorien tidligere at der opstår en harmonisk svingning når rødderne i karakterligningen er komplekse. 
Derfor er man interesserede i at få en god nok kombination af dæmpningskonstant og fjederkonstant til at rødderne i karakterligningen ikke bliver komplekse. 

Vi vil derfor gerne prøve at kigge på karakterligningen. 
Vi får da en ligning $mr^2 + \xi r + k = 0$ hvilket giver os løsningerne (med andengradslignings løsningsformel):

\begin{equation}
r = \dfrac{-\xi \pm \sqrt{\xi ^2 - 4mk}}{2m}
\end{equation}


For at vi ikke får komplekse rødder skal vi altså have at størrelsen $\xi ^2 -4mk$ er ikke-negativ, og det skal altså gælde at $0 \leq \xi ^2 -4mk $ eller omskrevet at $4mk \leq \xi ^2 $.
Her fra er man dog nødt til at vide mere omkring biler og kunne foretage nogle eksperimenter for at kunne fortsætte. 
Vi ved nemlig intet om $\xi$ og om hvordan dæmpning skabes. 

I forhold til videoen, kan vi se at bilen lige pludselig svinger og at karakterligningen derfor må have komplekse løsninger.
Vi ved derfor at der er sket en ændring af $k,m$ og $\xi$ sådan at det ikke længere gælder at $4mk \leq \xi ^2 $.
Da bilens masse, $m$, formentlig ikke har ændret sig, må vi altså enten have at $k$ er blevet større eller at $\xi$ er blevet mindre. 
Oversat til fysik, har vi altså enten at fjederen er blevet mere hård og derfor har en højere fjederkonstant eller at dæmpningssystemet ikke fungerer som det skal.  

